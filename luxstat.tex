\documentclass[a4paper,
               10pt,
               fleqn]{article}

\author{Ervin Mazlagic}
\title{Paper Manual}

\usepackage{luxpaper}
\usepackage{luxtitle}
 
\begin{document}
\luxtitle{Statuten}                             % Dokumentenart
         {LuXeria Statuten}                     % Titel
         {Ervin Mazlagi\'c\\Michael Zihlmann}   % Autor
         {Adligenswil}                          % Ort
         {2012}                                 % Jahr

\tableofcontents
\newpage

%%%%%%%%%%


\section*{Präambel}
Um die Lesbarkeit des vorliegenden Textes zu
erleichtern, wird auf die Doppelformulierung weiblich /
männlich verzichtet. Selbstverständlich sind in der
verwendeten maskulinen Form Männer und Frauen inbegriffen.

\section{Name, Sitz und Bestand}

\subsubsection*{Artikel §1}
Unter dem Namen Lu\textbf{X}eria besteht ein Verein im
Sinne von Art. 60 ff. ZGB mit Sitz in Luzern. Der
Verein wurde wurde am 24.02.2010 gegründet, ist auf
unbestimmte Zeit angelegt und endet mit dessen Auflösung.
\begin{itemize}
\item Der Verein ist politisch und konfessionell neutral 
und unabhängig.
\item Der Sitz und der Gerichtsstand befindet sich in
Luzern.
\end{itemize}

\section{Zweck}

\subsubsection*{Artikel §2}
Der Verein bezweckt die Förderung des Erfahrungsaustausches 
zwischen Technikinteressierten, sowie die Pflege der
Kameradschaft unter den Mitgliedern.\newline \newline
Folgendes sind die Hauptaktivitäten des Vereins:
\begin{itemize}
\item Einsatz von OpenSource-Betriebssystemen wie Linux und
 quelloffenen Applikationen fördern.
\item Intressierten Personen den Einstieg in die Benutzung 
von quelloffenen Systemen zu ermöglichen und zu erleichtern.
\item Austausch von technischen Informationen.
\item Durchführung von gemeinsamen Projekten.
\item Wöchentliches Treffen für den Erfahrungsaustausch.
\end{itemize}

\section{Mitgliedschaft}

\subsubsection*{Artikel §3}
Der Verein besteht aus Aktivmitgliedern, Passivmitgliedern, 
Ehrenmitgliedern und Gönnern. Als Mitglieder können alle
natürlichen und juristischen Personen aufgenommen werden.

\subsection{Aufnahme}

\subsubsection*{Artikel §4}
Die Mitgliedschaft wird mit schriftlicher oder mündlicher 
Beitrittserklärung und durch Bezahlen des
Mitgliederbeitrages erworben.

\subsection{Erlöschen der Mitgliedschaft}

\subsubsection*{Artikel §5}
Die Mitgliedschaft erlischt
\begin{itemize}
\item bei natürlichen Personen durch Austritt, Ausschluss 
oder Tod
\item bei juristischen Personen durch Austritt, Ausschluss 
oder Auflösung
\item automatisch mit der Auflösung des Vereins
\end{itemize}

\subsubsection*{Artikel §6}
Der Austritt ist durch schriftliche Erklärung an den 
Vorstand auf Ende des Kalenderjahres möglich.

\subsection{Ausschluss}

\subsubsection*{Artikel §7}
Mitglieder, deren Verhalten den Statuten widerspricht oder
den Vereinszwecken abträglich ist oder die ihren
Mitgliederbeitrag trotz Mahnung nicht bezahlen, werden durch
den Vorstand aus dem Verein ausgeschlossen. Der Ausschluss
muss vom Vorstand nicht weiter begründet oder kommentiert
werden.

\subsection{Aktivmitglieder}

\subsubsection*{Artikel §8}
Alle Interessierten können Aktivmitglied des Vereins werden.

\subsubsection*{Artikel §9}
Jedes Aktivmitglied hat bei Abstimmungen eine Stimme. 
Vertretungen sind nicht möglich.

\subsection{Passivmitglieder}

%\subsubsection*{Artikel §10}
%\textit{Dieser Artikel ist in den Statuten nicht vorhanden (Quelle: www.luxeria.ch)}

\subsubsection*{Artikel §10}
Passivmitglieder sind natürliche oder juristische Personen, 
welche die Mitgliedschaft wünschen und den jährlichen, von
der Mitgliederversammlung beschlossenen
Passivmitgliederbeitrag bezahlen.

\subsubsection*{Artikel §11}
Passivmitglieder werden zu den Mitgliederversammlungen 
eingeladen, haben jedoch kein Stimm- und Wahlrecht.

\subsection{Ehrenmitglieder}

\subsubsection*{Artikel §12}
Ehrenmitglieder sind Personen, welche sich durch persönliche
oder finanzielle Leistungen für den Verein besonders
eingesetzt haben.

Sie können von einer Mitgliederversammlung auf Antrag 
ernannt beziehungsweise aberkannt werden. Sie sind den
Aktivmitgliedern gleichgestellt, bezahlen aber keinen
Aktivmitgliederbeitrag.

\subsection{Gönner}

\subsubsection*{Artikel §13}
Gönner bzw. Sponsoren sind alle Personen, Organisationen 
oder Firmen, welche die Ziele des Vereins akzeptieren und
den Verein finanziell unterstützen möchten.

\section{Organe}

\subsubsection*{Artikel §14}
Die Organe des Vereins sind:
\begin{itemize}
\item die Mitgliederversammlung
\item der Vorstand
\end{itemize}

\subsection{Mitgliederversammlung}

\subsubsection*{Artikel §15}
Das oberste Organ des Vereins ist die Mitgliederversammlung.
 Die ordentliche Mitgliederversammlung findet jährlich
jeweils im ersten Quartal statt.

Die Einladung an die Mitglieder und Gönner erfolgt 
schriftlich mit der Traktandenliste mindestens 30 Tage im
voraus.

Über Geschäfte, die in der Traktandenliste nicht aufgeführt 
sind, muss kein Beschluss gefasst werden.

\subsubsection*{Artikel §16}
Jedes Mitglied hat das Recht, Anträge zuhanden der nächsten 
Mitgliederversammlung zu stellen.

Derartige Anträge sind in die Traktandenliste aufzunehmen, 
sofern sie dem Vorstand schriftlich und begründet bis 20
Tage vor der Versammlung zugestellt wurden.

\newpage
\subsubsection*{Artikel §17}
Die Mitgliederversammlung hat die folgenden Aufgaben:
\begin{itemize}
\item Wahl des Präsidenten, des Vorstandes sowie des 
Rechnungsrevisors.
\item Festsetzung und Änderung der Statuten.
\item Abnahme der Jahresrechnung und des Revisorenberichtes.
\item Kenntnisnahme des Jahresberichtes des Präsidenten 
oder des Vorstandes.
\item Entlastung des Vorstandes, des Kassiers und der 
Rechnungsrevisoren.
\item Beschlussfassung über das Jahresbudget.
\item Festsetzung der Mitgliederbeiträge.
\item Behandlung von Anträgen des Vorstandes oder von 
Mitgliedern sowie der Ausschlussrekurse.
\item Auflösung des Vereins.
\end{itemize}

\subsubsection*{Artikel §18}
An der Mitgliederversammlung besitzt jedes Aktive Mitglied 
eine Stimme. Die Beschlussfassung erfolgt mit einfachem Mehr
der anwesenden Stimmberechtigten. 
Der Präsident bringt im Falle eines Gleichstandes den 
Stichentscheid.

\subsection{Vorstand}

\subsubsection*{Artikel §19}
Der Vorstand besteht aus mindestens vier gewählten 
Aktivmitgliedern.

Er besteht namentlich aus:
\begin{itemize}
\item Präsident
\item Vizepräsident
\item Kassier
\item Aktuar
\end{itemize}
 
\subsubsection*{Artikel §20}
Der Vorstand wird für die Dauer von einem Jahr gewählt und 
ist nach Ablauf derselben wieder wählbar.

Er ist berechtigt, in der Zwischenzeit entstandene Vakanzen 
bis zur nächsten Mitgliederversammlung provisorisch zu
besetzen. Die Vorstandsmitglieder sind ehrenamtlich tätig
und haben grundsätzlich nur Anspruch auf Entschädigung ihrer
effektiven Spesen und Barauslagen für den Verein.

\section{Finanzen}

\subsection*{Artikel §21}
Vorstandsmitlieder können im Laufe des geltenden 
Kalenderjahres über Finanzmittel in Höhe des zweifachen
Mitgliederbeitrages frei verfügen. Überschreitungen sind
durch eine Mitgliederversammlung zu genehmigen.

\subsection{Unterschrift}

\subsubsection*{Artikel §22}
Für den laufenden Geldverkehr zeichnet der Präsident oder 
der Kassier mit Einzelunterschrift.

\subsection{Rechnungswesen}

\subsubsection*{Artikel §22}
Das Rechnungsjahr fällt mit dem Kalenderjahr zusammen. Für
den Geldverkehr ist ein Post- und/oder Bankkonto zu
eröffnen. Der Kassier führt die Buchhaltung.

\subsubsection*{Artikel §23}
Der Verein finanziert sich aus folgenden Mitteln:
\begin{itemize}
\item Jahresbeiträge der Mitglieder
\item Spenden und Gönnerbeiträge
\item Erlöse aus Aktionen, Projekten und Veranstaltungen
\item Erträge des Vereinsvermögens
\end{itemize}
 
\subsubsection*{Artikel §24}
Die Mitgliederversammlung legt die Höhe der Jahresbeiträge 
für die Aktiv- und Passivmitglieder fest.

\subsection{Mitgliederbeiträge}

\subsubsection*{Artikel §25}
Die Mitgliederbeiträge sind innerhalb von 60 Tagen nach der 
Mitgliederversammlung zu bezahlen.

Bezahlte Mitgliederbeiträge werden bei Austritt oder 
Ausschluss nicht zurück erstattet.

\subsection{Haftung}

\subsubsection*{Artikel §26}
Der Verein haftet nur mit dem Vereinsvermögen. Eine
persönliche Haftung der Mitglieder ist ausdrücklich
ausgeschlossen.

\subsection{Vereinsvermögen}

\subsubsection*{Artikel §27}
Austretende oder ausgeschlossene Mitglieder haben keinen 
Anspruch auf Teile des Vereinsvermögens.

\section{Statuten}

\subsubsection*{Artikel §28}
Die Statuten sind von der Gründungs- oder 
Mitgliederversammlung zu genehmigen.

Abänderungsvorschläge müssen schriftlich vorliegen.
Zur Genehmigung von Statutenänderungen ist die absolute
Mehrheit der anwesenden, stimmberechtigten Mitglieder
erforderlich.

\section{Auflösung des Vereins}

\subsubsection*{Artikel §29}
Die Auflösung des Vereins kann nur durch eine
ausschliesslich zu diesem Zweck einberufene,
ausserordentliche Mitgliederversammlung beschlossen werden,
an der mindestens zwei Drittel der Aktivmitglieder anwesend
sind.

Trifft dies nicht zu, so ist innerhalb von vier Wochen eine 
zweite Mitgliederversammlung einzuberufen, die ohne
Rücksicht auf die Zahl der anwesenden Mitglieder
beschlussfähig ist.

\section{Schlussbestimmungen}

\subsubsection*{Artikel §30}
Im Falle der Liquidation muss das Vereinsvermögen einer oder
 mehreren gemeinnützigen Organisationen, welche den gleichen
oder einen ähnlichen Zweck verfolgen überwiesen werden. Im
Falle dass keine Einigung zu Stande kommt, wird das
Vereinsvermögen der Gemeinde oder dem Amt überwiesen, in 
welcher sich der Sitz des Vereins befindet.

Die ausserordentliche Mitgliederversammlung legt die Details
dieses Beschlusses fest.

\section{Inkrafttreten}
Diese Statuten sind per Zirkularbeschluss vom
\today~angenommen worden und sind mit diesem Datum in Kraft
getreten.\newline\newline


\begin{tabular}{l r l}
\indent \textbf{Präsident} & 
    \indent \indent \indent \indent \indent \indent &
        \textbf{Vizepräsident} \\
\indent Ervin Mazlagic     & 
    \indent \indent \indent \indent \indent \indent &
        Michael Zihlmann \\
\end{tabular}


%%%%%%%%%%

\end{document}
