\section{Mitgliedschaft}

\subsubsection*{Artikel §\articlenumber}
Der Verein besteht aus Aktivmitgliedern, Passivmitgliedern, Ehrenmitgliedern
und Gönnern. Als Mitglieder können alle natürlichen und juristischen Personen
aufgenommen werden.

\subsection{Aufnahme}

\subsubsection*{Artikel §\articlenumber}
Die Mitgliedschaft wird mit schriftlicher oder mündlicher Beitrittserklärung
und durch Bezahlen des Mitgliederbeitrages erworben.

\subsection{Erlöschen der Mitgliedschaft}

\subsubsection*{Artikel §\articlenumber}
Die Mitgliedschaft erlischt
\begin{itemize}
	\item bei natürlichen Personen durch Austritt, Ausschluss 
		oder Tod
	\item bei juristischen Personen durch Austritt, Ausschluss 
		oder Auflösung
	\item automatisch mit der Auflösung des Vereins
\end{itemize}

\subsubsection*{Artikel §\articlenumber}
Der Austritt ist durch schriftliche Erklärung an den Vorstand auf Ende des
Kalenderjahres möglich.

\subsection{Ausschluss}

\subsubsection*{Artikel §\articlenumber}
Mitglieder, deren Verhalten den Statuten widerspricht oder den Vereinszwecken
abträglich ist oder die ihren Mitgliederbeitrag trotz Mahnung nicht bezahlen,
werden durch den Vorstand aus dem Verein ausgeschlossen. Der Ausschluss muss
vom Vorstand nicht weiter begründet oder kommentiert werden.

\subsection{Aktivmitglieder}

\subsubsection*{Artikel §\articlenumber}
Alle Interessierten können Aktivmitglied des Vereins werden.

\subsubsection*{Artikel §\articlenumber}
Jedes Aktivmitglied hat bei Abstimmungen eine Stimme. Vertretungen sind nicht
möglich.

\subsection{Passivmitglieder}

%\subsubsection*{Artikel §10}
%\textit{Dieser Artikel ist in den Statuten nicht vorhanden
%(Quelle: www.luxeria.ch)}

\subsubsection*{Artikel §\articlenumber}
Passivmitglieder sind natürliche oder juristische Personen, welche die
Mitgliedschaft wünschen und den jährlichen, von der Mitgliederversammlung
beschlossenen Passivmitgliederbeitrag bezahlen.

\subsubsection*{Artikel §\articlenumber}
Passivmitglieder werden zu den Mitgliederversammlungen  eingeladen, haben
jedoch kein Stimm- und Wahlrecht.

\subsection{Ehrenmitglieder}

\subsubsection*{Artikel §\articlenumber}
Ehrenmitglieder sind Personen, welche sich durch persönliche oder finanzielle
Leistungen für den Verein besonders eingesetzt haben.

Sie können von einer Mitgliederversammlung auf Antrag ernannt beziehungsweise
aberkannt werden. Sie sind den Aktivmitgliedern gleichgestellt, bezahlen aber
keinen Aktivmitgliederbeitrag.

\subsection{Gönner}

\subsubsection*{Artikel §\articlenumber}
Gönner bzw. Sponsoren sind alle Personen, Organisationen oder Firmen, welche
die Ziele des Vereins akzeptieren und den Verein finanziell unterstützen möchten.
