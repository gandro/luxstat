\section{Organe}

\subsubsection*{Artikel §\articlenumber}
Die Organe des Vereins sind:

\begin{itemize}
	\item die Mitgliederversammlung
	\item der Vorstand
\end{itemize}

\subsection{Mitgliederversammlung}

\subsubsection*{Artikel §\articlenumber}
Das oberste Organ des Vereins ist die Mitgliederversammlung. Die ordentliche
Mitgliederversammlung findet jährlich jeweils im ersten Quartal statt.
Die Einladung an die Mitglieder und Gönner erfolgt schriftlich \change{per
E-Mail oder Brief} mit der Traktandenliste mindestens 30 Tage im voraus.
Über Geschäfte, die in der Traktandenliste nicht aufgeführt sind, muss kein
Beschluss gefasst werden.

\subsubsection*{Artikel §\articlenumber}
Jedes Mitglied hat das Recht, Anträge zuhanden der nächsten
Mitgliederversammlung zu stellen.

Derartige Anträge sind in die Traktandenliste aufzunehmen, sofern sie dem
Vorstand schriftlich und begründet bis 20 Tage vor der Versammlung zugestellt
wurden.

\newpage
\subsubsection*{Artikel §\articlenumber}
Die Mitgliederversammlung hat die folgenden Aufgaben:

\begin{itemize}
	\item Wahl des Präsidenten, des Vorstandes sowie des
		Rechnungsrevisors.
	\item Festsetzung und Änderung der Statuten.
	\item Abnahme der Jahresrechnung und des Revisorenberichtes.
	\item Kenntnisnahme des Jahresberichtes des Präsidenten oder des
		Vorstandes.
	\item Entlastung des Vorstandes, des Kassiers und der
		Rechnungsrevisoren.
	\item Beschlussfassung über das Jahresbudget.
	\item Festsetzung der Mitgliederbeiträge.
	\item Behandlung von Anträgen des Vorstandes oder von Mitgliedern
		sowie der Ausschlussrekurse.
	\item Auflösung des Vereins.
\end{itemize}

\subsubsection*{Artikel §\articlenumber}
An der Mitgliederversammlung besitzt jedes Aktive Mitglied eine Stimme. Die
Beschlussfassung erfolgt mit einfachem Mehr der anwesenden Stimmberechtigten. 
Der Präsident bringt im Falle eines Gleichstandes den Stichentscheid.

\subsection{Vorstand}

\subsubsection*{Artikel §\articlenumber}
Der Vorstand besteht aus mindestens vier gewählten Aktivmitgliedern.
Er besteht namentlich aus:

\begin{itemize}
	\item Präsident
	\item Vizepräsident
	\item Kassier
	\item Aktuar
\end{itemize}
 
\subsubsection*{Artikel §\articlenumber}
Der Vorstand wird für die Dauer von einem Jahr gewählt und ist nach Ablauf
derselben wieder wählbar.
Er ist berechtigt, in der Zwischenzeit entstandene Vakanzen bis zur nächsten
Mitgliederversammlung provisorisch zu besetzen. Die Vorstandsmitglieder sind
ehrenamtlich tätig und haben grundsätzlich nur Anspruch auf Entschädigung
ihrer effektiven Spesen und Barauslagen für den Verein.

\subsubsection*{Artikel §\articlenumber}
\change{Der Vorstand hat das Recht eine Abstimmung einzuberufen um Entscheide
vom Verein zu Bestätigen.}
\change{Ein per Abstimmung gefällter Entscheid ist gleichwertig zu Beschlüssen
einer ordentlichen Mitgliederversammlung.}
\change{Eine Abstimmung erfolgt schriftlich per E-Mail oder Brief und hat eine
minimale Behandlungsfrist von vier Wochen.}
\change{Bei Abstimmungen besitzt jedes aktive Mitglied eine Stimme. Die
Beschlussfassung erfolgt mit einfachem Mehr der Stimmabgaben.}
