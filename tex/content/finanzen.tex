\section{Finanzen}

\subsection*{Artikel §\articlenumber}
Vorstandsmitlieder können im Laufe des geltenden 
Kalenderjahres über Finanzmittel in Höhe des zweifachen
Mitgliederbeitrages frei verfügen. Überschreitungen sind
durch eine Mitgliederversammlung zu genehmigen.

\subsection{Unterschrift}

\subsubsection*{Artikel §\articlenumber}
Für den laufenden Geldverkehr zeichnet der Präsident oder 
der Kassier mit Einzelunterschrift.

\subsection{Rechnungswesen}

\subsubsection*{Artikel §\articlenumber}
Das Rechnungsjahr fällt mit dem Kalenderjahr zusammen. Für
den Geldverkehr ist ein Postkonto und/oder Bankkonto zu
eröffnen. Der Kassier führt die Buchhaltung.

\subsubsection*{Artikel §\articlenumber}
Der Verein finanziert sich aus folgenden Mitteln:
\begin{itemize}
\item Jahresbeiträge der Mitglieder
\item Spenden und Gönnerbeiträge
\item Erlöse aus Aktionen, Projekten und Veranstaltungen
\item Erträge des Vereinsvermögens
\end{itemize}
 
\subsubsection*{Artikel §\articlenumber}
Die Mitgliederversammlung legt die Höhe der Jahresbeiträge 
für die Aktiv- und Passivmitglieder fest.

\subsection{Mitgliederbeiträge}

\subsubsection*{Artikel §\articlenumber}
Die Mitgliederbeiträge sind innerhalb von 60 Tagen nach der 
Mitgliederversammlung zu bezahlen.
Bezahlte Mitgliederbeiträge werden bei Austritt oder 
Ausschluss nicht zurück erstattet.

\subsection{Haftung}

\subsubsection*{Artikel §\articlenumber}
Der Verein haftet nur mit dem Vereinsvermögen. Eine
persönliche Haftung der Mitglieder ist ausdrücklich
ausgeschlossen.

\subsection{Vereinsvermögen}

\subsubsection*{Artikel §\articlenumber}
Austretende oder ausgeschlossene Mitglieder haben keinen 
Anspruch auf Teile des Vereinsvermögens.
